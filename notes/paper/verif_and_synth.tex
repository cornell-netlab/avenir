\section{Verification and Synthesis}

\begin{figure*}
  \begin{tabular}{l l c c r >{$ \Leftrightarrow$}c l}
    \toprule
    && Problem
    & Assumption
    & \multicolumn{3}{c}{Validity Condition} \\ \midrule
    \multirow{3}{*}{\rotatebox[origin=c]{90}{\it Basic}} 
    && $\textsc{Verif}(\tau_l, \tau_r)$
    & n/a
    & $\wp_{\tau_l}(c_l)$ && $\wp_{\tau_r}(c_r)$ \\
    && $\textsc{InstSynth}(\tau_l) = \tau_r$
    & n/a
    & $\wp_{\tau_l}(c_l)$ && $\wp_{\tau_r}(c_r)$ \\
    && $\textsc{MapSynth}() = F$
    & n/a
    & $\forall \tau_l. \wp_{\tau_l}(c_l)$ && $\wp_{f(\tau_l)}(c_r)$ \\ \midrule
    \multirow{4}{*}{\rotatebox[origin=r]{90}{\it Edit-based}}
    && $\textsc{EditSynth}() = f$
    & $\wp_{\tau_l}(c_l) \Leftrightarrow \wp_{\tau_r}(c_r)$
    & $\forall \rho. \wp_{(\tau_l \otimes \rho)}(c_l)$ && $\wp_{\left(\tau_r\otimes \bigotimes f(\rho)\right)}(c_r)$ \\
    && $\textsc{EditSynth}(\rho) = f$
    & $\wp_{\tau_l}(c_l) \Leftrightarrow \wp_{\tau_r}(c_r)$
    & $\wp_{(\tau_l \otimes \rho)}(c_l)$ && $\wp_{\left(\tau_r\otimes \bigotimes f(\rho)\right)}(c_r)$ \\
    \cmidrule(r){2-7}
    % & \textsc{EditSynth}$(\oplus, s)$
    % & n/a
    % & $\wp(c_l\; \tau_l) \Leftrightarrow \wp(c_r\; \tau_r)$
    % & $f : \mathsf{Edit}_s \to 2^{\mathsf{Edit}}$
    % & $\forall \rho_{s}. \wp(s^\oplus\parenbb{c_l}_{(\tau_l \oplus \rho_s)})$ \\
    % &&&&\qquad$\Leftrightarrow \wp\left(\hole\parenbb{c_r}_{\left(\tau_r\otimes \bigotimes f(\rho_s)\right)}\right)$ \\
    & \multirow{3}{*}{\rotatebox[origin=r]{90}{\it Bound}}
    & $\textsc{EditCheck}[n]()$
    & $\wp_{\tau_l}(c_l) \Leftrightarrow \wp_{\tau_r}(c_r)$
    & $\forall \rho_s. \exists \rho_1, \ldots, \rho_n. \wp(\parenbb{c_l}_{\tau_l}^\rho)$
    &&$\wp(\parenbb{c_r}_{\tau_r}^{\rho_1,\ldots,\rho_n})$ \\
    && $\textsc{EditSynth}[n]() = f_1,\ldots, f_n$
    & $\wp_{\tau_l}(c_l) \Leftrightarrow \wp_{\tau_r}(c_r)$
    & $\forall \rho. \wp(\parenbb{c_l}_{\tau_l}^\rho))$ && $\wp(\parenbb{c_r}_{\tau_r}^{f_1(\rho),\ldots,f_n(\rho)})$ \\
    && $\textsc{EditVerif}[n](f_1,...,f_n)$
    & $\wp_{\tau_l}(c_l) \Leftrightarrow \wp_{\tau_r}(c_r)$
    & $\forall \rho. \wp(\parenbb{c_l}_{\tau_l}^\rho))$ && $\wp(\parenbb{c_r}_{\tau_r}^{f_1(\rho),\ldots,f_n(\rho)})$\\ \addlinespace
    \bottomrule
  \end{tabular}
  \caption{Cheat Sheet for the Verification and Synthesis problems.\textit{ All
      problems take $c_r$ and $c_l$ as given. Additional givens are in
      parentheses.  Problem parameters are given in square brackets. Note,
      $F : \mathsf{Inst} \to \mathsf{Inst}$, $f : \Edit \to 2^\Edit$ and
      $f_i : \Edit \to \Edit$ }}
  \label{fig:problems}
\end{figure*}

We can use this calculus to construct some verification and synthesis
problems. Broadly we want to take a logical program $c_l$ and a concrete program
$c_r$, relate their respective table instantiations $\tau_l$, and $\tau_r$,
where $\tau_i$ is $c_i$-complete.  We generate our verification conditions using
Dijkstra's weakest precondition semantics, included for completeness in
Figure~\ref{fig:wp}.

\paragraph{Verification} The simplest problem is the verification problem
\textsc{Verif} where we are given the table instantiations $\tau_l$ and
$\tau_r$, and we want to verify that $c_l\; \tau_l$ is equivalent to
$c_r\; \tau_r$. Since everything is concrete, we can encode this as a simple
\textsc{sat} problem. If the following problem returns \texttt{unsat}, we
conclude $c_l\; \tau_l$ and $c_r\; \tau_l$ are equivalent.
\[\begin{array}{lcl}
    \text{Given }( c_l\; \tau_l), (c_r\; \tau_r) \\
    \exists \fvs(c_l), \fvs(c_r). \\
    \qquad \neg(wp(c_l\; \tau_l) \iff wp(c_r\; \tau_r))
  \end{array}
\]

\paragraph{Synthesizing a Concrete Instance} One step up, we have the
\textsc{InstSynth} problem, where only the logical table instantiation is given,
and we must produce a $\tau_r$ satisfying the \textsc{Verif} condition. If we
get \texttt{sat} and a model for $\tau_r$ from Z3 we conclude success.
\[\begin{array}{lcl}
    \text{Given } c_l\; \tau_l,c_r
    \exists \tau_r.\\
    \forall \fvs(c_l), \fvs(c_r). \\
    \qquad wp(c_l\; \tau_l) \iff wp(c_r\; \tau_r)
  \end{array}
\]

In a real network, this means that every time the
controller changes the logical forwarding state, we must resynthesize the
concrete forwarding state. However, network engineers care very deeply about the
predictability of their networks, and inserting a heuristic shim in the middle
of the network would make it very hard to exert the kind of control that network
engineers rely on.

\paragraph{Synthesizing an Instance Mapping} To ameliorate this problem, we consider the \textsc{MapSynth} problem, which
synthesizes a function $f$ from an arbitary logical instantiations $\tau_l$ to a
concrete table instantiation that will implement the same functionality as
$\tau_l$. The offline synthesis succeeds the following formula is \texttt{sat}
and the witness for $f$ is the solution.
\[\begin{array}{l}
    \text{Given } c_l,c_r\\
    \exists f. \forall \tau_l. \forall \fvs(c_l), \fvs(c_r). \\
    \qquad wp(c_l\; \tau_l) \iff wp(c_r\; f(\tau_l))
  \end{array}\]
However this function $f$ is really quite large, and for even trivial
examples causes Z3 to return \texttt{unknown}. So, we need to find a way of
partitioning the \textsc{OfflineSynth} problem into more tractable
components.

\paragraph{Edit-Based Synthesis} Instead of trying to synthesize an arbitrary
function from table instantiations to table instantiations, we reframe the
problem to examining only the table edits, and call it \textsc{EditSynth}. The
idea is: given two equivalent programs, for every edit to a logical table, can
we synthesize an equivalent set of edits to the concrete tables. This means we
synthesize (offline!) two different functions for each table, one for additions
and one for deletions. This works so long as we have a notion of transactions.

\todo[inline]{The following functions and objects need types!}


\begin{figure}[pt]
  \[\rho \in \Edit = \{ +, -\} \times \Name \times \Expr^n \times \mathbb N \]
  \[\begin{array}{lcl}
      \tau \otimes (+,s,\vec e, i) & \triangleq & \tau[s \mapsto (e,i)::\tau(s)] \\
      \tau \otimes (-,s,\vec e, i) & \triangleq & \tau[s \mapsto \tau(s) \setminus (e,i)] \\
      \tau + (+, s, \vec, e, i) & \triangleq & \tau \otimes (+,s, \vec e,i) \\
      \tau - (-, s, \vec, e, i) & \triangleq & \tau \otimes (-, s, \vec e, i) 
    \end{array}\]

  \[\begin{array}{lcl}
      \rho_s & \triangleq & (\oplus, s', \vec e, i), \text{where } s = s'\\
      \Edit_s &\triangleq & \{\rho_s \mid \rho_s\in \Edit\} \\
    \end{array}\]
  \caption{Edits}
  \label{fig:edits}
\end{figure}

Formally, an edit $\rho \in \mathsf{Edit}$ is a tuple
$(\oplus,s,\vec e,i)$, where $\oplus \in \{+,-\}$, $s$ is the name of
the table being edited, and $(\vec e, i)$ is the row to add. Then,
given a table instantiation function $\tau$, we write
$\tau \otimes \rho$ to update $\tau$ with $\rho$. Precisely,
$\tau \otimes (+,t,\vec e, i) = \tau[t \mapsto (e,i)::\tau(t)]$, and
$\tau \otimes (-, t, \vec e, i) = \tau[t \mapsto \tau(t) \setminus
(e,i)]$ where $l \setminus (e,i)$ removes all occurences $(e,i)$ from
the list $l$. As syntactic sugar, we'll write $\tau + \rho$ to mean
$\tau \otimes \rho$ when $\rho = (+, s, \vec e, i)$, and $\tau - \rho$
to mean $\tau \otimes \rho$ when $\rho = (-, s, \vec e,
i)$. Additionally, we adopt the convention that $\rho_s$ indicates an
edit to all tables named $s$, i.e. that
$\rho_s = (\oplus, s, \vec e, i)$. Similarly, $\mathsf{Edit}_s$ is the
set of all such edits. Edits are summarized in Figure~\ref{fig:edits}.



A solution for \textsc{EditSynth} is a witness $f$ for every table $t$ such that
the following formula is satisfied:
\[\begin{array}{l}
    \text{Given } c_l,c_r, \exists \rho. \\
    \forall \tau_l,\tau_r,\rho. \forall \fvs(c_l), \fvs(c_r). \\
    \qquad \wp(c_l\; \tau_l) \iff wp(c_r\; \tau_l) \\
    \qquad \Rightarrow \wp(c_l\; (\tau_l \otimes \rho)) \iff \wp(c_r\; \tau_r \otimes (f(\rho)))
  \end{array}\]

For now we have only figured out how to address a simpler version of the
problem:

A solution for \textsc{EditSynth'} is a series of actions $\vec{\rho'}$ for every table $t$ such that
the following formula is valid:
\[\begin{array}{l}
    \text{Given } c_l,c_r, \\
    \forall \rho, \tau_r, \tau_l.\, \exists n.\, \exists \hole\rho_1,\ldots,\hole\rho_n. \\ 
    \quad \forall \fvs(c_l), \fvs(c_r). \\
    \qquad \wp(c_l\; \tau_l) \iff wp(c_r\; \tau_l) \\
    \qquad \Rightarrow \wp(c_l\; (\tau_l \otimes \rho)) \iff \wp(c_r\; \tau_r
    \otimes \hole\rho_1 \otimes \cdots \otimes \hole\rho_n )
  \end{array}\]

In practice, we will can even produce problems \textsc{EditSynth}$(\oplus,s)$
where the universal $\rho$ is restricted so that its first two elements are
$\oplus$ and $s$.