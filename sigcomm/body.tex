\lstset{language=C}
\lstset{
 morekeywords={action,table,control}
}

\section{Introduction}

\todo[inline]{write short introduction}

\section{The Problem}
\todo[inline]{Explain problem of Control Plane Synthesis with example }

\section{Synthesis by Sketching is a Solution}
\todo[inline]{Explain GCL/WP/CEGIS/Sketching formalisms at a very how
  it can be applied to P4 programs }

\todo[inline]{Conclude that this monolithic solution is too slow}

Sketching~\cite{armando-thesis,armando2013sketch} is a program synthesis technology that 
takes high-level insights about the desired system and automatically infers the details.
The input to the synthesis procedure is a partial program (sketch) that has {\em holes}
in place of code fragments.
The task of the synthesizer is to fill up the holes with the right pieces so that the 
result program meets the required specification.  
The algorithm to generate code for the holes of a sketch is counterexample-guided inductive synthesis (CEGIS).
To describe sketching and CEGIS in the context of our problem domain we use the running 
example of Figure~\ref{fig:running-p4}.

\begin{figure}
\centering
\begin{minipage}[t]{.23\textwidth}
\centering
\begin{framed}\hspace{-.5em}
\begin{lstlisting}[basicstyle=\ttfamily\footnotesize, lineskip = 0em,numbersep=0pt,resetmargins=true, xleftmargin=-1em]
action _smac(v){smac=v}
action _dst(v){dst=d}
action _out(v){out=v}
table L1 {
  keys = {src:exact}
  actions =
    {_smac; _dst}
}
table L2 {
  keys = {dst:exact}
  actions = {_out}
}
control Ingress {
  L1.apply();
  L2.apply()
}
\end{lstlisting}
\end{framed}
(a)
\end{minipage}
\begin{minipage}[t]{.23\textwidth}
\centering
\begin{framed}\hspace{-1em}
\begin{lstlisting}[basicstyle=\ttfamily\footnotesize, lineskip = 0em,numbersep=0pt,resetmargins=true, xleftmargin=-1em]
action _smac_out(v1,v2)
  { smac=v1;out=v2 }
action _dst_out(v1,v2)
  { dst=v1;out=v2 }
action _out(v) {out=v}
table P {
  keys = { src:exact
         ; dst:exact }
  actions =
   { _smac_out
   ; _dst_out
   ; _out }
}
control Ingress {
  P.apply()
}
\end{lstlisting}
\end{framed}
(b)
\end{minipage}

\begin{ttfamily}
\begin{footnotesize}
\setlength{\columnsep}{0pt}
\begin{multicols}{3}
\begin{tabular}{c|c}
 {\ttfamily src} & action \\\hline
       1         & smac := 1 \\
       2         & dst := 1 \\
       *         & no-op
\end{tabular}

\columnbreak

\begin{tabular}{c|c}
 {\ttfamily dst} & action \\\hline
       1         & out := 1 \\
       2         & out := 2 \\
       *         & no-op
\end{tabular}

\columnbreak

\begin{tabular}{c|c|c}
 {\ttfamily src} & {\ttfamily dst} & action \\\hline
                 &                 &        \\
                 &                 &        \\
\end{tabular}
\end{multicols}
\end{footnotesize}
\end{ttfamily}

\vspace*{-4pt}
\centering (c)


\caption{P4 programs of (a) logical and (b) physical packet processors, (c) current installed rules\label{fig:running-p4}}
\end{figure}

\begin{figure}
\begin{tabular}{c@{\hskip 1.5em}|@{\hskip 1.5em}c}
\begin{small}
$
        \begin{array}{l}
          \IF \\
          \phantom{\IF}~\var{src} = 1 \to \var{smac} := 1 \\
          \phantom{\IF}~\var{src} = 2 \to \var{dst} := 1\\
          \phantom{\IF}~\var{src} = \var{*} \to \SKIP \\
          \ENDIF ; \\
          \IF \\
          \phantom{\IF}~\var{dst} = 1 \to \var{out} := 1 \\
          \phantom{\IF}~\var{dst} = 2 \to \var{out} := 2 \\
          \phantom{\IF}~\var{dst} = \var{*} \to \SKIP \\
          \ENDIF
        \end{array}
$
\end{small}
&
\begin{small}
$
        \begin{array}{l}
          \IF~\\
          \phantom{\IF}~\var{src} = \var{*} \wedge \var{dst} = \var{*}\to \SKIP \\
          \ENDIF 
        \end{array}
$
\end{small}
\\
(a)
&
(b)
\end{tabular}

\caption{Gaurded command encoding for (a) logical and (b) physical P4 programs with their current rules\label{fig:encoding}}
\end{figure}

\begin{figure}
  \fbox{
  \begin{minipage}[t]{0.53\columnwidth} 
  \[
    \begin{array}{>{\hspace{-.5em}}lcll}
      c &::=&\\
        & \mid & \SKIP & \mathit{Skip} \\
        & \mid & x := e & \mathit{Assign} \\
        & \mid & t.\apply & \mathit{Table} \\
        & \mid & c;c & \mathit{Sequence} \\
        & \mid & \IF~\overline{b \to e}~\ENDIF & \mathit{Select} \\
      t & ::=  & (s, \overline{f}, \overline{c}, c)  \\
      s & \in & \multicolumn 2 l {\TableName} \\
      x & \in & \multicolumn 2 l {\Field \cup \Metadata} \\
      v & \in & \multicolumn 2 l {\{0,1\}^+ = \BitVector}
    \end{array}
  \]
\end{minipage} \vline \begin{minipage}[t]{0.45\columnwidth}
  \[
    \begin{array}{lcll}
      e & ::= & \\
        & \mid & v & \mathit{Value} \\
        &  \mid & x & \mathit{Variable} \\
        & \mid & e + e & \mathit{Plus}\\
        & \mid & e - e & \mathit{Minus}\\
        & \mid & e \mask e & \mathit{Mask}\\
      b & ::= & \\
        & \mid & 0 & \mathit{Zero}\\
        & \mid & b \Rightarrow b  &\mathit{Implies}\\
        & \mid & e = e &\mathit{Equals} \\
        & \mid & e < e & \mathit{Less}
    \end{array}
  \]
\end{minipage}}
\caption{The syntax of packet processing pipelines}
\label{fig:syntax}
\end{figure}


The logical program in Figure~\ref{fig:running-p4}(a) has a pipeline of two match-action 
tables ({\ttfamily L1} and {\ttfamily L2}).
The first table {\ttfamily L1} matches on {\ttfamily src} and sets the packet headers
{\ttfamily smac} and {\ttfamily dst}.
Table {\ttfamily L2} matches on {\ttfamily dst} and sets {\ttfamily out}.
The physical program in Figure~\ref{fig:running-p4}(b) has a single unified table {\ttfamily P}
that matches on {\ttfamily src}, {\ttfamily dst} and sets $\{${\ttfamily smac}, {\ttfamily out}$\}$
and $\{${\ttfamily dst}, {\ttfamily out}$\}$.
Figure~\ref{fig:running-p4}(c) shows the set of rules in the logical tables
(for the sake of simplicity the fields are integer numbers).
A challenging question is to determine if there exist a set of rules for the physical table 
so that after installation of such rules the physical and logical network 
behave exactly the same on every input packet.
To address this challenge, we first convert the tables with their installed rules into programs
in guarded command programming language~\cite{dijkstra1975}.

Figure~\ref{fig:syntax} shows the syntax of the Guarded Command Language (GCL) 
that we use in this paper.
A single guarded command rule is of the form $\IF~{b \to e}~\ENDIF$ where $b$ is a Boolean
condition and $c$ is a command.
There can be multiple guarded commands inside an $\IF$.
When there are more than a single guard equivalent to {\it true},
we pick the first guard and execute its corresponding command 
(note that this interpretation is different from the classical semantics where 
a satisfying guard is picked randomly).
Figure~\ref{fig:encoding} shows the encoding of the logical and physical programs into 
guarded commands.
Assuming that {\ttfamily src}, {\ttfamily dst}, {\ttfamily smac} and {\ttfamily out}
are variables in the program, each guarded command is responsible for a single row 
of the table.
Note that the physical table does not have any rule so its guarded command does not 
change any variable.

We may wonder if the logical and physical programs of Figure~\ref{fig:encoding}
are already equivalent for any incoming packet so there is no need to insert any extra rules.
Formal verification, an important constituent of any synthesis procedure, answers such equivalence queries.
A synthesizer uses formal verification to decide if the current candidate solution is equivalent 
to what the user actually wants.
In the case that the synthesized program is not the right solution, verification stops with a counter-example to show
the reason for unequivalence.
To use formal verification, we need a formal way to represent the semantics of the GCL programs.
In this paper we use the axiomatic semantics~\cite{hoare1969} which uses logic to give meaning to a program.

The basic idea in the axiomatic approach is to use pre- and post-conditions to show the 
meaning of a command $c$.
If the pre-condition $P$ is met, executing the command $c$ ensures that the post-condition $Q$ holds.
In the Hoare triple notation, this is shown as: $\{P\}~c~\{Q\}$.
We often formulate axiomatic semantics using predicate transformation: given a pre-condition (or post-condition)
transform the condition through the program to find the post-condition (or pre-condition).
For example, let's take the first rule of the table {\ttfamily L1} in the logical table.
If we consider the most general condition on the variable {\ttfamily smac} after the
execution of the rule ($x$ is a fresh variable), what is correct the pre-condition on the variables?

\begin{center}
$
\{?\} ~~ \IF~{\mbox{\ttfamily src} = 1 \to \mbox{\ttfamily smac} := 1}~\ENDIF ~~ \{\mbox{\ttfamily smac} = x\}
$ 
\end{center}


In general, there can be many valid pre-conditions for this Hoare triple.
One valid pre-condition can be $\mbox{\ttfamily src} = 2$ which is a packet that
does not take the rule (since {\ttfamily src}$\neq1$) 
so it does not put any restrictions on the general post-condition $\mbox{\ttfamily smac} = x$.
However, such a pre-condition is not general enough and shows only a specific case.
In describing semantics we are normally looking for the most general (or weakest)
pre-conditions for a command. 
In this example, the weakest pre-condition is $\mbox{\ttfamily src}=1\rightarrow x = 1$
which basically says that {\ttfamily src} can get any value but when it is equal to 1
then the value of $x$ must be equal to $1$ as well (since the rule is taken).
For computation of weakest pre-conditions we use a set of deductive rules.
Figure~\ref{fig:table-wp} shows the weakest pre-condition rules for GCL.
To show that the programs in Figure~\ref{fig:encoding} are equivalent or not,
the formal verifier computes the weakest-preconditions (using the given rules) 
of the both programs with respect to a general packet with fresh symbolic fields.
Only if the two conditions are equal for any values of the variables we 
conclude that the programs are equivalent.
The verifier consults a theorem prover to check the equivalence of the these two universally quantified formulae.
Since the programs in Figure~\ref{fig:encoding} are not equivalent, 
the verification engine will return a counter-example packet to show that the 
programs behave differently.
For example, a counter-example is $\{\mbox{\ttfamily src}=0,\mbox{\ttfamily dst}=0\}$.
When inserted into the logical program, the matched actions will update the header values to 
$\{\mbox{\ttfamily out}=1,\mbox{smac}=1\}$.
In the physical program the header values stay the same since there is no rule in the table.

Verification is one of the main modules in the CEGIS algorithm.
The other one is the inductive synthesizer module as Figure~\ref{fig:cegis} shows.
When verification fails and the verifier produces a concrete counter-example,
it passes the counter-example to the inductive synthesis module.
Inductive synthesizer creates a candidate solution 
(in this case an instantiation of the physical program)
that works correctly for the given counter-example.
The main idea in CEGIS is to make several iterations on verification and 
inductive synthesis until a valid solution is produced.
CEGIS is guaranteed to stop and return a solution in a finite domain but 
it may take several iterations.
We will describe some of our heuristics to reduce the number of iterations 
in the later sections.

\begin{figure}
\begin{tikzpicture}[auto,>=latex']
    \node [draw] (is) {Inductive Synthesis};
    \node [draw, right =2cm of is] (verif) {Verification};
    \node [ellipse,draw, right=1em of is] (f) {Fail};
    \node [ellipse,draw, right=1em of verif] (o) {Ok};

    \draw [->] (is) -- (f);
    \draw [->] (verif) -- (o);
    \draw [->,bend left=40] (verif) node[below=1em] {Fail} to node{counterexample} (is);
    \draw [->,bend left=40] (is)  to node{candidate implementation} (verif);
\end{tikzpicture}
\caption{CEGIS Loop\label{fig:cegis}}
\end{figure}


Synthesizer creates a candidate solution for the given input with the help of sketching.
A sketch in our problem domain is a guarded command with holes in place of some constants.
For removing the counter-example $\{\mbox{\ttfamily src}=0,\mbox{\ttfamily dst}=0\}$, 
the user may provide the following sketch to the synthesizer.
This sketch is assisting the synthesis procedure by saying that 
the required rule should match on the guards $\mbox{\ttfamily src}=0\wedge\mbox{\ttfamily dst}=0$
and the the holes $?_0$, $?_1$, $?_2$ and $?_3$ in the the head of the implication 
are unknown.

\[
\begin{aligned}
 &\IF \\
 &~\mbox{\ttfamily src}=0\wedge\mbox{\ttfamily dst}=0 \to \\
 &~~~~\mbox{\ttfamily src}=?_0~;~\mbox{\ttfamily out}=?_1~;~\mbox{\ttfamily smac}=?_2~;~\mbox{\ttfamily dst}=?_3 \\
 &\ENDIF
\end{aligned}
\]

Note that in principle the user can also leave holes in place of $0$ in the guard.
This only makes the space of search for the synthesizer larger.
We do not require smart sketches from the user in our tool but providing them can 
make the process more efficient.


To find the correct constants for the holes, we again compute the weakest pre-condition
of a packet with respect to the given sketch.
This time we do not consider a general packet with fresh header values.
The goal here is to find values for the holes in a way that the behavior of 
the physical network includes the given example.
Synthesizer computes weakest-precondition for the specific packet that 
the verifier passes to it (in this case 
\{\mbox{\ttfamily src$=0$},\mbox{\ttfamily dst$=0$},\mbox{\ttfamily out$=1$},\mbox{\ttfamily smac$=1$}\}).
The holes in the {\it wp} computation are converted to existentially quantified variables.
After computing the weakest pre-condition, the synthesizer again consults a theorem prover.
If theorem prover determines that the formula is unsatisfiable, 
then the synthesizer announces that synthesis is impossible and the CEGIS loop stops.
Otherwise, if the formula is satisfiable, theorem prover provides a model for the formula.
Model assigns constant values to the existentially quantified variables.
In the case of the sketch above, the holes $?_1$ and $?_2$ get the value $1$ and the theorem prover
do not give any values for $?_0$ and $?_3$  (they are irrelevant for this example packet).

The basic procedure above can be very time-consuming since it takes a monolithic approach 
(i.e. models every rule from every table).
When there are multiple rules in the table,
the {\it wp} formula will contain a disjunction that grows with the number of rules.
This can lead to exponential blow-up when the physical network contains 
a pipeline of tables.
To address this issue, we use compact verification condition generation~\cite{flanagan2001}, 
slicing (Section ??), path abstraction (Section ??) and widening 
when there are ranges in the problem (Section ??).
Each of these techniques can drastically reduce the required time 
to solve benchmarks.


% Example

% - cross product
% - set-read dependencies
% % - backtracking
% % - action data

% a in A and b in  B
% A must set y sometimes

% two logical tables 
% (, src, A,skip).apply();(dst, B, skip).apply()

% A = {(\v -> smac := v)
%   + skip
%   + (\v -> dst := v)}

% B = {(\v -> out := v)
%   + skip}

% L = {(\v1 -> \v2 -> smac := v1; out := v2)
%   + (\v -> smac:=v)
%   + skip
%   + (\v1 -> \v2 -> dst := v1; out:= v2)
%   + (\v -> dst:= v) }

% src |   A         dst | B
% ---------------; -----------
% 1     smac:= 1    1      out :=1
% 2     dst := 1    2      out :=2
% *     skip        *     skip
% ------>

% Logical table encoding : spec
% if
% src = 1 ---> smac := 1
% src = 2 --> dst := 2
% *  ----> skip
% fi;
% if
% dst = 1 ----> out := 1
% dst = 2 ----> out := 2
% *       ---> skip
% fi

% one physical table
% (p, (src,dst), L, skip).apply()

% src  dst  | L
% -------------------
% *     *      skip

% physical sketch:  sketch
% if
% src = ?src /\ dst = ?dst
% ->
%   if
%   (?select_p = 0) ->  smac := ?v1; out := ?v2
%   (?select_p = 1) -> (smac := ?v)
%   .....
%   fi
% * -> skip

% phi = //\\ { x = x' | x in hdrs, x' is fresh}

% sketch' = assign random values to holes in sketch
% Equivalence:
% wp(spec, phi) <=> wp(sketch', phi)
% | Valid -> done
% | CE pkt -> 
% Model-finding:
% [|spec|](pkt) = pkt'
% wp(sketch, pkt') = phi
% ask z3 find model for holes:
% \exists holes  \forall hdrs. pkt /\ phi
% | UNSAT -> fail ``sorry go home''
% | Sat model ->  
% Update:
%   convert ?keys -> keys
%   convert ?select_p -> action choice
%   convert ?vs -> action data

% go to equivalence  
  
\section{Challenges}
\todo[inline]{Describe the technical challenges with this approach with forward pointers to solutions}
\begin{itemize}
\item Programs are big, but control plane APIs are incremental
\item CEGIS loop is clunky
  \begin{itemize}
  \item Many paths to search through
    \begin{itemize} \item candidate map \end{itemize}
  \item One logical rule may correspond to many physical rules
    \begin{itemize} \item symbolic execution to compute membership VCs \end{itemize}
  \item VCs are big
    \begin{itemize}
    \item cormac
    \item hybrid solvers
    \end{itemize}
  \end{itemize}
Abduction -- when does our loop work?
  \begin{itemize}
  \item conditions on action usability
  \item ??dataflow??
  \end{itemize}
\end{itemize}

\section{A Scalable Solution}
\todo[inline]{High level description of the Solution}

\subsection{Syntax}

Our syntactic representation of packet-processing pipelines can be
found in Figure~\ref{fig-syntax}. Notice that it is more or less the
same as the syntax for GCL, except for the additional table
application command $t.\apply$, which indicates that the controller
can install rules that obey the schema $t$ into this table. A table
schema $t$ is a tuple $(s, \overline f, \overline c, d)$, where $s$ is
a unique identifer for the table, $\overline f$ is the sequence of
fields that the table will match on, $\overline c$ is the set of
actions that can be applied, and $d$ is the default action. Just like
in P4, actions are a subset of full commands, namely, they can only be
assignments and sequences.

Our expression sublanguage focuses solely on bitvector arithmetic and
masking operations. The simplest expression is a bitvector literal. We
can write down variables $x$ that are used to represent packet fields
or metadata. We can combine two expressions expressions $e$ and $e'$
into more complicated ones using unsigned arithmetic addition
$e + e'$, unsigned arithmetic subtraction $e - e'$, or bitwise
intersection $e \mask e'$. We could easily add additional arithmetic
and bitvector operators, but these operators suffice for this
exposition.

The language of booleans is standard. We can express falsehood ($0$),
implication ($\Rightarrow$), bitvector equality $(=)$, and comparison
($<$). In what follows, we will use the standard syntactic sugar
encodings of the remaining boolean and comparison operators freely.

\subsection{Semantics}
\begin{figure*}
  \[\begin{array}{lcl}
      \denote{\SKIP}^\tau~(\pkt,\meta)
      &\triangleq
      & (\pkt,\meta) \\
      \denote{f := e}^\tau(\pkt,\meta)
      &\triangleq
      & (\pkt\{f \mapsto \denote{e}(\pkt,\meta)\}, \meta)\\
      \denote{m := e}^\tau(\pkt, \meta)
      &\triangleq
      & (\pkt, \meta\{f \mapsto \denote{e}^\tau(\pkt,\meta)\})\\
      \denote{c_1;c_2}^\tau(\pkt, \meta)
      &\triangleq
      & \denote{c_2}\left(\denote{c_1}^\tau(\pkt, \meta)\right)\\
      \denote{\IF~\ENDIF}^\tau(\pkt, \meta)
      & \triangleq
      & (\pkt, meta) \\
      \denote{\IF~\overline{b \to c}~\ENDIF}^\tau (\pkt,\meta)
      & \triangleq & \begin{cases}
        \denote{c_1}^\tau(\pkt,\meta)& \denote{b_1}^\tau(\pkt, \meta) = 1 \\
        \denote{\IF~b_2 \to c_2 \cdots b_n \to c_n~\ENDIF}^\tau (\pkt,\meta) & \mathit{otherwise}
      \end{cases} \\
      \denote{t.\apply}^\tau(\pkt,\meta)
      & \triangleq
      & \run(\tau(s),\overline f, c_0)(\pkt,\meta) \\\\
      \run(\cdot, \overline f, \overline c, c_0)(\pkt, \meta)
      & \triangleq
      & \denote{c_0}^\tau(\pkt,\meta) \\
      \run(((\overline k, a)\cdot\overline e), \overline f, \overline c,  c_0) (\pkt, meta)
      & \triangleq
      & \begin{cases}
        \denote{c_a}^\tau(\pkt, \meta) & \denote{\overline k = \overline f} (\pkt, \meta) = 1 \\
        \run(\overline e, \overline f,\overline c, c_0) (\pkt, \meta) & \mathit{otherwise} \\
        \end{cases}
    \end{array}
  \]
  \caption{Semantics of pipeline processing programs. The denotations
    of bitvector and boolean expressions are omitted for brevity}
\end{figure*}

In a formal setting, we can model the denotation of packet processing
programs as functions on packets ($\Pkt \to \Pkt$). However, our
commands are not fully executable on their own -- we need to know how
to populate the tables before we know what function the programs
denote. We record the existing table rules in a \emph{table
  instantiation function},
$\tau : \TableName \rightharpoonup \Entry^*$, which maps the name of
table, to a list of the entries that have been inserted by the
controller. A valid entry $(k_1,\ldots, k_n, i) \in \Entry$ in a table
$t$ with keys $f_1, \ldots, f_m$, actions $a_1, \ldots, a_l$ and
default action $a_0$, has $n = m$, and $i \in \{0, \ldots, l\}$, where
each $k_i \in \BitVector$.
\todo[inline]{Incorporate inexact matches}

We also need to consider the packet metadata, which we will encode as
a separate packet, and will not be considered in packet equality.

Most of the cases of our denotational semantics are standard. The
denotation $\SKIP$ is the identity function; assignment $x := e$
updates the metadata $\meta$ or the packet $\pkt$ depending on whether
$x \in \Metadata$ or $x \in \Field$; sequence $c_1;c_2$ first executes
$c_1$ and then executes $c_2$; and selection
$\IF~\overline{b \to c}~\ENDIF$ iterates through the list of guards
$b$, and executes the first $c_i$ for which $b_i$ holds true.

The semantics of table application
$(s, \overline f, \overline c, d).\apply$ is a bit nonstandard, but
very similar to the semantics of selection. We iterate through the
rows $\overline{(\overline k, i)}$ of $\tau(s)$, executing the first
$a_i$ for which
$\denote{\overline f = \overline k}^\tau~\pkt~\meta = 1$. If no such
entry exists then the default action is chosen. Note that the
semantics for a table application are undefined if $\tau(s)$ is
undefined.

The semantics of expressions and booleans are standard, although we
make the assumption the expressions are ``well-typed'', i.e. the
expressions on either side of the operators denote vectors of the same
length. This property is easy to statically check.

\subsection{Edits}

However, the controller doesn't interface with a switch, or a table,
in terms of full instantiations, interacts with the tables in terms of
incremental insertions. There are three kinds of operations that a
controller can take: \emph{insertion}, \emph{deletion}, and
\emph{update}. For the purposes of this paper, we only focus on
insertions, since deletions and assertions require the same mechanisms
as insertion does.

We write an edit $e \in \Edit$ as a pair $(s, \rho)$ with
$s \in \TableName$ and $\rho \in \Entry$, which denotes the insertion
of row $\rho$ into table $s$. Given a table instantiation function
$\tau$, we can write $\tau + e$ to mean the map
$\tau[s \mapsto \tau(s) \append \rho]$. Note that the edit inserts a
row at the end of the sequence of entries, which means it has the
lowest priority (ahead of only the default action).

\todo[inline]{Say something about priority insertion}

\subsection{Predicate Transformer Semantics}

\begin{figure}
  \[
    \begin{array}{lcl}
      \wp_\tau(\SKIP,\phi)
      & \triangleq
      & \phi \\
      \wp_\tau(x := e, \phi)
      & \triangleq
      & \phi[e/x]\\
      \wp_\tau(c_1;c_2, \phi)
      & \triangleq
      & \wp_\tau(c_2, \wp_\tau(c_1, \phi)) \\
      \wp_\tau(\IF~\overline{b \to c}~\ENDIF, \phi)
      & \triangleq
      & (\bigvee_i b_i) \wedge \\
      && \bigwedge_i\left(b_i \wedge \bigwedge_{j=1}^{j-1}\neg b_j\right) \Rightarrow \wp_\tau(c_i, \phi)\\
      \wp_\tau(t.\apply, \phi)
      &\triangleq
      &  \bigwedge_{(\overline{k_i}, i) \in \tau(s)}\hit\left(\tau(s), \overline f, i\right)\Rightarrow \wp(c_i,\phi) \\
      && \wedge \allmiss(\tau,s) \Rightarrow \wp(c_0, \phi) \\
      && \textit{where } t = (s, \overline f, \overline c, c_0) \\
      \hit\left(\overline{\left(\overline{k}, a\right)},\overline f, i\right)
      &\triangleq& \overline{k_i} = \overline f \wedge \bigwedge_{0<j<i} \overline k \neq \overline f\\
      \allmiss\left(\tau,s\right) &\triangleq& \bigwedge_{\left(\overline k, a\right) \in \tau(s)} \overline k \neq \overline f
    \end{array}
  \]
  \caption{Weakest Precondition for pipelines}
  \label{fig:table-wp}
\end{figure}

The CEGIS algorithm relies on a way to generate boolean formulae that
capture the behavior of the programs. As in the standard CEGIS
approach, we compute the weakest precondition of a symbolic packet to
denote the input-output relation denoted by the command. However, just
as in the denotational semantics, we need a table instantation in
order to describe the behavior of tables. In previous work~\cite{pv4},
tables have been implemented as a n-ary nondeterministic demonic
choice between the actions. While this certainly covers all possible
behaviors of the table, it throws away information about the keys --
which is necessary knowledge for our synthesis problem.

Unlike~\cite{p4v}, we are doing our verification and synthesis with
our table instances $\tau$ in hand, which gives us the power to
describe more precisely the behavior of tables in our logical
formulae. This is captured in the function $\wp_\tau(c, \phi)$
described in Figure~\ref{fig:table-wp}. When we compute the weakest
precondition of table $t = (s, \overline f, \overline c, c_0)$ of a
formula $\phi$ with respect to an instantiation
$\tau(s) = (\overline{k_1}, a_1), \ldots, (\overline{k_n}, a_n)$, we
separately compute the condition $\hit(\tau, s, i)$ under which each
entry $i$, and the weakest precondition of executing that action
$\wp_\tau(c_{a_i}, \phi)$. Then we intersect
$\hit(\tau, s, i) \Rightarrow \wp(c_{a_i}, \phi)$ for every $i$. Then
for the default action, we compute $\allmiss(\tau, s)$, which just
negates the match condition of every entry, as the guard for
$\wp_\tau(c_i, \phi)$.

The predicate $\hit(\tau, s, i)$ must be true if entry $i$ is hit.
The entry will be hit iff the the keys match
$(\overline{k_i} = \overline f)$ and if none of the previous keys do
$\bigwedge_{j<i}\overline {k_j} \neq \overline f$. So
$\hit(\tau, s, i)$ is just the conjunct of these two predicates.

The predicate $\allmiss(\tau, s , i)$ could be defined as
$\bigwedge_i \neg(\hit(\tau, s, i))$, but we can define it
equivalently, and much more simply as
$\bigwedge_i \overline {k_i} \neq \overline f$.


\subsection{Incremental Synthesis}

The monolithic Sketching alorithm was designed for general-purpose
programs, and as such, is designed to be run once, in an offline
fashion, to statically synthesize a performant program. A direct port
of this algorithm will try to recompute all of physical rules in the
logical program upon every rule insertion. To witness this, imagine
the controller executes a sequence of rules insertions
$e_1,\ldots, e_n$, which corresponds to a sequence of logical table
instantiation functions $\tau_1, \ldots, \tau_n$. Then, the monolithic
CEGIS algorithm will synthesize a corresponding sequence of physical
table instantiations $\tau_1', \ldots, \tau_n'$.

This seems wasteful. Each $\tau_{i+1} = \tau_i + e_{i+1}$, and as
defined, an insertion can only append a single rule to the bottom of a
single table. And in many cases $\tau_{i+1} = \tau_i +
\overline{e'}$. Why not leverage the fact that we already had two
instatiation functions that induced equivalent functions? All we need
to do is synthesize the deltas!

\todo[inline]{describe translation here or in earlier CEGIS Section}

Instead of asking the CEGIS loop to recompute every insertion in
$\tau_i$, we can ask it to only compute the sequence of additions in
$\overline{e}$. Instead of instrumenting the physical program with a
completely blank slate, we use the previous $\tau'$ and only add holes
necessary to compute the insertions. We insert 3 categories of holes
into each table $(s, \overline f, \overline c, c_0)$:
\begin{itemize}
\item (Necessity) Add a binary hole $\mathit{?AddTo}_s$ that is true
  when we need to insert a rule into that table
\item (Action Choice) For every action, add a sequence of guards
  $\mathit{?Act}_s = i$, following the command $c_i$.
\item (Keys) For each key $f_i$, add holes $\mathit{?f}_i$.
\end{itemize}

Together these three holes correspond to a new row in the select
corresponding\todo{define this somewhere!} to table $s$ of the form
\[
  \begin{array}l
    \overline{f} = \overline{\mathit{?f}} \wedge \mathit{?AddTo}_s = 1
    \to \left(\begin{array}l \IF \\
          \qquad \mathit{?Act}_s = 1 \to c_1 \\
          \qquad \cdots\\
          \qquad \mathit{?Act}_s = n \to c_n \\
          \ENDIF
        \end{array} \right)
  \end{array}
\]

This way, when z3 gives us a model for these holes, we know that we
must insert a rule into every table $s$ for which corresponding to
$\mathit{?AddTo}_s$ is true, and that rule will have the form
$(\overline{\mathit{?f}}, \mathit{?Act}_s)$.

\todo[inline]{example}

\todo[inline]{Conflict step!}

\subsection{Model-Finding Optimizations}
\begin{itemize}
\item Get models using paths
\item Candidate map
\item Generalizing counter examples for range queries
\item Program Slicing
\item caching rule templates
\item Linear-size VCs
\end{itemize}

\subsubsection{Path Abstraction}
\label{sec:path-abstraction}
When a packet $\pkt$ enters a switch, it will ``see'' exactly one
sequence of assignments. That is, absent cloning and multicast, each
packet will take either the true or false branch of each if statement,
and hit exactly one rule in a table. We call this sequence of
assignments (and the guards required to execute them) a \emph{trace}
$\sigma(\pkt) \in \Cmd$. A trace is defined formally in
Figure~\ref{fig:traces}.

However, in our CEGIS loop, we compute the weakest precondition of the
hole-ified physical program with respect to the \emph{full
  program}. This includes the paths that execute completely different
funcitonality or are unreachable by the input packet. However, we know
that the counter-example input packet $\pkt_0$ will take exactly one
path through both the logical and physical programs, so we can instead
search through the paths through the hole-ified physical program for
paths that could possibly implement the path that $\pkt_0$ took
through the logical program, i.e. the weakest precondition of the
output counterexample packet $\pkt_1$ is not equivalent to $\FALSE$.

With physical-path weakest precondition $\phi$ of $\pkt_1$ in hand we
ask Z3 for a model satisfying
$\forall x_i. x_1 = \pkt.x_1 \wedge \ldots x_n = \pkt.x_n \Rightarrow
\pkt_1$, which corresponds to either 0 or 1 edits to the physical
program.

\subsubsection{Precomputed Candidate Map}
\label{sec:candidate-map}
An observation made in the NetCore~\cite{} work is that the greatest
difficulty in compiling networking programs to switches is in
compiling the classifier that is used to determine which actions
should be executed; the action languages are simple enough that this
check is often syntactic or nearly-so.

We find that we can reduce the search space (either by shrinking the
full program, or by reducing the size of the set of paths described in
Section~\ref{sec:path-abstraction}) by precomputing the set of
physical action-traces that could possibly implement any given action
trace in the logical program.

To do this, we first compute (via cross-product) the set of possible
logical action traces $\Tr(log)$, by simply iterating through all
choices of the actions in the tables, including the default action. In
order to capture action data, we simply replace bound variables with
fresh holes. Then we do the same for the physical program $phys$ to
create $\Tr(phys)$.

Now, we need to consider when an action trace can be implemented in
terms of another. Without action data, this reduces to equivalence of
straight-line programs. However, in their presence, we need to abduce
constraints on the action data that can be used. If this abduced
condition is equivalent to unsat (which check via Z3) we simply do not
include that action-trace as a canidate for the logical action-trace.

We bundle these observations into a map
$\alpha : \Tr(log) \to 2^{\Test \times \Tr(phys)}$ such that
$(\phi, \sigma_p) \in \alpha(\sigma_l)$ whenever $\phi$ implies that
$\sigma_p = \sigma_l$. Now, once we enter the inductive synthesis
phase, and observe a counterexample, we can symbolically execute the
input-packet counterexample $\pkt_0$ through the logical program and
observe corresponding action-trace $\sigma_l$. Now, when we're trying
to find a model, the path-based abstraction of
Section~\ref{sec:path-abstraction} lets us simply skip over the
logical traces that do not contain a sequence in $\alpha(\sigma_l)$ as
a subsequence. We expect this optimization to be effective since, in
the common case, $|\alpha(\sigma_l)|$ is close to 1, whereas the
number of paths through a switch program may number in the tens of
thousands.


\subsubsection{Automatic Widening}

A counter-example input packet $\pkt_0$ takes a path $\sigma(\pkt_0)$,
we assume that an input packet's invalidity is indicative of the
invalidity of the whole path. So, instead of exploring the whole space
of packets that traverse $\sigma(\pkt_0)$, one-by-one, we will attempt
to synthesize a rule (or rules) that solve for a whole class of
counter-example packets.

\todo[inline]{Something about heuristic widening templates?}

To do this, we will change the way that we instrument our physical
program. Instead of sketching equality conditions ($x = ?x$ for every
key $x$ in a table), we can sketch range conditions:
$?x_{l}\leq x \leq ?x_{h}$ for every key $x$ in a table. These ranges,
of course, subsume equality conditions, so we do not lose any
generality so far.

However these holes are sometimes too general. Observe that our
model-finding queries have the form

\[\exists \overline{?x_h},\overline{?x_l}.~\forall \bar x.~\bigwedge x = \pkt_0.x \Rightarrow \phi\]
where $\phi$ is the instrumented precondition of a trace or
program. Assuming that we have only one field, $x$, in both the
logical and physical programs, and that $\pkt_0.x = 7$, then
we want to solve the formula
\[\exists ?x_h,?x_l. \forall x. x = 7 \Rightarrow ?x_l \leq x \leq ?x_h\]
for which there are many unfounded solutions, such as $?x_l = 0$ and
$?x_h = 10000$. To constrain our solutions to the domain of the
counterexample, we will use Z3's MaxSat solver to mininimize the size
of the interval that we synthesize, which means that for the above
formula, our model is $?x_l = 7 = ?x_h$.

However, this doesn't get us any more expressive power! Leveraging the
observation that the behavior of $\pkt_0$ is indicative of all the
behavior on its path, we want the assumption of our condition to be
weakest precondition that a packet must satisfy to take that path:
$\wp(\sigma(\pkt_1), \TRUE)$. We need to do a similar thing in the
logical program to avoid polluting our formula with
unnecessarily-specific equality conditions latent in the formula
constructed by $\pkt_1$; for each trace
$\tau \in \alpha(\sigma(\pkt_1)$, we then search for a model
(minimizing $\sum (?x_h - ?x_l)$) that satisfies
\[\exists \overline{?x_l},\overline{?x_h}. \forall \overline
  x. \wp(\sigma(\pkt_1),\TRUE) \Rightarrow \wp(\tau, \TRUE)\]

\subsection{Verification Optimization}
One of the biggest bottlenecks in our CEGIS algorithm is the
evaluation of the equivalence query. Using standard the
weakest-precondition functions, the size of the verification condition
(VC) we compute is exponential in the size of the input
program~\cite{dijkstra}. This section explores ways to mitigate this blowup.

\subsubsection{Linear-Size VCs}
Packet processing pipelines never fail. Packets are dropped or proceed
normally. Differing from other work which encodes packet
drops similarly to $\assert \FALSE$, we include a special
\texttt{drop}-bit into our programs that encodes whether packets are
dropped, which mimics the physical reality of many
architectures.

Prior work has shown that you can compute quadratic-size logical
characterizations of programs in so-called \emph{passive} form to a
program transformation that reduces eliminates duplications latent in
Dijkstra's weakest preconditions. A direct corollary of their work is
that if your programs are assertion-free, as ours are, then you can
compute \emph{linear-size} verification conditions.

However, because, like in SSA form, passive programs give a different
index to every variable, our relation program equivalence cannot be
captured by $N(log) \Leftrightarrow N(phys)$, because $N(log)$ will
capture all the different intermediate values that the fields of $log$
go through (sim $N(phys)$), and we are only concerned with equivalence
of the initial and final states. We also must rename the variables $x$
in $phys$ to fresh names $x'$ to avoid capture, we call this new
program $phys'$. To capture this we record the initial index ($0$) and
the final index ($\max(x)$) of a variable $x$, to construct the
following verification condition:
\[\begin{array}{l}
    N(log) \wedge N(phys') \wedge \bigwedge_{x \in \fvs(log;phys)} x_0 = x'_0 \\
    \Rightarrow \bigwedge_{x \in fvs(log;phys)} x_{\max(x)} = x'_{\max(x)}
  \end{array}
\]

However, this is too strong to handle dropped packets -- if both
programs drop the packet, we don't care what its final value is, only
that they both dropped the packet. We modify our verification
condition to its final form below:
\[ \begin{array}{l}
     N(log) \wedge N(phys') \wedge \bigwedge_{x \in \fvs(log;phys)} x_0 = x'_0  \\
     \Rightarrow \texttt{drop}_{\max(\texttt{drop})} = 0 = \texttt{drop}'_{\max(\texttt{drop}')}\\
     \phantom{\Rightarrow}~\vee  \bigwedge_{x \in fvs(log;phys)} x_{\max(x)} = x'_{\max(x')}
   \end{array}
 \]

Note that for small programs this VC is larger than the one computed
via weakest preconditions, however, the linear scaling rapidly
compensates for this fact.

\subsubsection{Heuristic Initial Counter-Examples}
To kick off our synthesis procedure on a given insertion, we first
check whether we need to make any changes at all to the physical
network -- it could be that the rule had no end-to-end semantic
effect. If our new logical program and the physical program have
diverged, Z3 will tell us and give us a counter example packet.

Assume for the moment that the insertion changes the behavior of the
program (we'll discuss the alternate case below). However, whatever
counter-example Z3 produces (if it does at all) will necessarily hit
the rule we just added! So instead of producing a possible
counterexample by checking equivalence of two programs, we instead can
produce a counterexample by asking Z3 for a packet that will hit the
new rule we installed. This formula will necessarily be smaller since
it only concerns a portion of one program, not two full programs.

Once we produce a counter example packet $\pkt$, we can check whether
it has different behavior in both the logical and physical programs
making it a true counter example. If it is a false CE, we can default
back to the full verification check; but if it is a true CE we can go
directly to the incremental synthesis phase.

\subsubsection{Program Slicing}




\subsection{Abduction}
\begin{itemize}
\item cross-product examination of paths
\item Dataflow analysis
\end{itemize}

\subsection{Theoretical Results}
\begin{itemize}
\item Soundness
\item Completeness of Synthesis
\item Completeness of Abduction
\end{itemize}

\section{Implementation}
\todo[inline]{Describe the Implementation}
\begin{itemize}
\item P4 $\to$ Pipeline (including ternary explosion)
\item Pipeline + Instance $\to$  GCL
\end{itemize}

\subsection{Limitations}
\begin{itemize}
\item equivalent parsers
\item registers
\item worst-case blowup from ternary
\end{itemize}

\section{Evaluation}
\todo[inline]{Describe the experimental setup and analyze results}
\subsection{Real World Examples from ONF's Trellis}
\subsection{Synthetic Examples}
\subsection{Benchmark Abduction}

\section{Related Work}
\todo[inline]{Write Related Work} \todo[inline]{Sketch, p4v, Synthesis
  for P4, domino, panopticon, control plane stuff for openflow}
\todo[inline]{other synthesis stuff?}

\section{Conclusion}
\todo[inline]{Write Conclusion}

