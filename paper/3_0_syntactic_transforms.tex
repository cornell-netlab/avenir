\subsection{Action-Directed Synthesis}

Observe that in the previous formulations, all of the higher-orderness
comes from the unknown match conditions. What if, \emph{a la} \texttt{p4v} we
just elminiate the matches, and model tables as a demonic choice
between all of the actions. We formalize this notion in
Figure~\ref{fig:nondeterministic}, using $|c|$ to represent this
encoding.

\begin{figure}[ht]
  \[\begin{array}{l >{\triangleq}c l}
      |x := e| && x := e  \\
      |c;c'| && |c|;|c'| \\
      |c \angel c'| && |c|\angel |c'| \\
      |\assert b | && \assert b \\
      |\assume b | && \assume b \\
      |\apply {t, \vec k, \vec{c}, c_d}| && h_1^t := 0; \cdots ; h_n^t := 0; h_d^t := 0; \\
      \multicolumn 2 l {} & \Bigg(\bigg(\displaystyle\Angel_{c_i \in \vec c} c_i; h_i^t := 1\bigg) \angel c_d; h^t_d := 1 \Bigg); \\
      \multicolumn 2 l {} & \assert {h^t_d \veebar (\exists!c_i. (h_j^t))} 
    \end{array} \]
  \caption{Nondeterministic Encoding of Tables}
  \label{fig:nondeterministic}
\end{figure}

Now we can define a problem \textsc{NonDetVerif} which is a very
coarse-grained check -- it rules out program pairs that have
incompatible sets of actions. Given two programs $c_l$ and $c_r$, we
can simply check the validity of the following:

\[\forall\pkt, \pkt'.\; \wp(|c_l|, \pkt = \pkt') \Leftrightarrow \wp(|c_l|, \pkt = \pkt') \]

\begin{example}
  Let
  $c_l = \apply{\texttt{ip}, (\texttt{pt}), \langle \texttt{ip.src} :=
    4 \rangle , \assert \TRUE }$, and let
  $c_r = \apply {\texttt{tcp}, (\texttt{pt}), \langle \texttt{tcp.src}
    := 4 \rangle, \assert \TRUE}$. Intuitively, This pair will not
  pass the \textsc{NonDetVerif} condition, because the \texttt{ip}
  table modifies the \texttt{ip.src} variable and the \texttt{tcp}
  table modifies the \texttt{tcp.src} variable. We compute
  $|c_l| = \texttt{ip.src}:=4 \angel \assert \TRUE$ and
  $|c_r| = \texttt{tcp.src} := 4 \angel \assert \TRUE$. So,
  $\wp(|c_l|) \Leftrightarrow \wp(|c_r|)$ is equivalent to
  $(\texttt{ip.src}' = 4 \wedge \texttt{tcp.src} = \texttt{tcp.src}')
  \Leftrightarrow (\texttt{ip.src}' = \texttt{ip.src}' \wedge
  \texttt{tcp.src}' = 4)$, which is clearly false in general.
\end{example}


The intuition here, is that \textsc{NonDetVerif} is a necessary, but
not sufficient condition for the above solutions,
i.e. $\textsc{MapSynth} \Rightarrow \textsc{NonDetVerif}$.

There is an easy generalization of this check to a synthesis problem,
we want to check which actions are feasible implementations of a given
action in a given table. We call this the \emph{feasible domain} of an
action-table pair. We again, need to come up with a different, but
related encoding of tables for a given action-table pair $(c_i,
s)$. This is defined in Figure~\ref{fig:hitenc}. The problem of
computing the a path that can implement an action is
\[\textsc{PathSynth}(c_i, s) = \left\langle h_j^t \mid c_j \in \actions t \cup \{\default t\},
    t \in \Tables {c_r}\right\rangle,\] which produces a collection of
bits $h_j^t$, which are set to true if action $c_j \in \actions t$ can
be used to implement action $c_i \in \actions s$. The set of solutions
to \textsc{PathSynth} is the feasible domain. Notice again that
\textsc{PathSynth} is a necessary, but insufficient condition -- if
there is no path, then \textsc{MapSynth} cannot succeed.

\begin{figure}[ht]
  \[\begin{array}{l>{\triangleq}cl}
      \mathcal H | x:= e | && x := e \\
      \mathcal H | c; c'| && \mathcal H |c|;\mathcal H |c'| \\
      \mathcal H | c \angel c'| && \mathcal H | c | ; \mathcal H | c' | \\
      \mathcal H | \assert b | && \assert b \\
      \mathcal H | \assume b | && \assume b \\
      \mathcal H | \apply{t, \vec k, \vec c, c_d} | && \assert {h_d^t \veebar \exists! c_i. h_i^t }; \\
      \multicolumn 2 l {} & \left(\displaystyle\Angel_{c_i}\assume {h_i^t};c_i \right) \angel \assume {h^t_d} ; c_d      
    \end{array}\]
  \caption{Instrumentation to for Feasible Domain Synthesis. The
    ($\exists!$) can be desugared in the obvious way since $\vec c$ is
    finite. This assumption ensures that the table is deterministic,
    i.e. the $h_i^t$s pick out a path.}
  \label{fig:hitenc}
\end{figure}

Then, we can define solutions to $\textsc{DomSynth}(c_i,s)$ by a
satisfying model to the following formula:
\[\begin{array}{l}\exists (h_i^t)_{\left(\substack{t \in \Tables{c_r} \\ c_i \in \actions t }\right)}. \;\forall \pkt. \; \forall \pkt'. \\
    \qquad \wp(\mathcal H |c_l|, \pkt = \pkt') \Leftrightarrow \wp(\mathcal H |c_r|, \pkt = \pkt') \end{array}\]


Now that we have a collection of the possible ways to encode the
actions, we need to make sure that our tests can be encoded
effectively. Morally, we want to know that a rule addition will only
impact the same set of equivalence classes that it would impact in the
logical program.
%
%
\[ \bracbb{c_l }_{\tau_l \otimes \rho}\; \pkt \neq \bracbb{c_l}_{\tau_l} \; \pkt \Leftrightarrow
  \bracbb{c_r}_{\tau_l \otimes \bigotimes f(\rho)} \neq \bracbb{c_r}_{\tau_l}\; \pkt \]


\begin{proposition}[Edit Completeness]
  Our axioms are sufficient to construct an $f : Edit \to \Edit^*$,
  given (logical and concrete) deterministic, total programs $c_l$ and
  $c_r$, such that for every pair of complete instantiations $\tau_l$
  and $\tau_r$, such that
  $\bracbb{c_l}_{\tau_l} = \bracbb{c_r}_{\tau_l}$, then, for every
  $c_l$-well-formed $\rho \in \Edit$ such that
  $\tau_l \neq \tau_l \otimes \rho$, then
  $\bracbb{c_l}_{\tau_l \otimes \rho} = \bracbb{c_r}_{\tau_l \otimes
    f(\rho)}$
\end{proposition}


\begin{proof}
  Let $\rho = (+, s, \vec e, i)$. Consider an input packet $\pkt$, and
  run $\bracbb{c_l}_{\tau_l}\; \pkt$. Since $c_l$ is deterministic and
  total, there is a table trace, $\sigma_l \in \Edit^*$ corresponding
  to the rows that $\pkt$ hits in $c_l$ instantiated with
  $\tau_l$. There is another table trace $\sigma_l^\rho$ corresponding
  to the tables hit by $\pkt$ when running
  $\bracbb{c_l}_{\tau_l \otimes \rho}\; \pkt$.

  Now there are two cases, either $\sigma_l = \sigma_l^\rho$, or not.



  Consider the case where
  $\bracbb{c_l}_{\tau_l}\; \pkt_0 = \bracbb{c_l}_{\tau_l \otimes \rho}
  \; \pkt_0$. Then,
  $\pkt = \pkt_0 \wedge \wp(\parenbb{c_l}_{\tau_l}, \pkt = \pkt')
  \Leftrightarrow \pkt = \pkt_0 \wedge \wp(\parenbb{c_l}_{\tau_l
    \otimes \rho}, \pkt = \pkt')$. Further, since
  $\tau_l \otimes \rho \neq \tau_l$, we know that $\pkt$ misses
  $\rho$\todo[inline]{formalize}


  We need to show that
  $\pkt = \pkt_0 \wedge \wp(\parenbb{c_r}_{\tau_r}, \pkt = \pkt')
  \Leftrightarrow \pkt = \pkt_0 \wedge \wp (\parenbb{c_r}_{\tau_r
    \otimes f(\rho)}, \pkt = pkt')$.


  There are a few situations we need to consider in this proof
  \begin{itemize}
  \item $\pkt_0$ hits $\rho$ and $\pkt_0$ hits a sub-trace of
    $f(\rho)$. This can be proven by restricting the programs to the
    respective actions.
    
  \item $\pkt_0$ hits $\rho$ and $\pkt_0$ misses all of
    $f(\rho)$. Contradictory if we assume $\rho$ is effectful.

  \item $\pkt_0$ misses $\rho$ and $\pkt_0$ hits a sub-trace of $f(\rho)$.
  \item $\pkt_0$ misses $\rho$ and $\pkt_0$ misses of $f(\rho)$. This
    can be proven by restricting the programs to miss the traces.
  \end{itemize}
  
\end{proof}


\subsection{Lets start small -- Single-table Switches}

Assume we can convert our programs into a single table with infinite
rows that allows choice in the actions (we can do it for a disjunction
thereof or if we add wild-cards, but for now just imagine a single
table). The problem becomes much simpler. We want to map one table
into another. How do we do this?

We need to map actions -- which we can do using \textsc{DomSynth} and
then make sure we can always discriminate them via the keys. Note that
in this context, \textsc{DomSynth} is picking out a single concrete
action for each logical action.

So now all we need to do is constrain \textsc{DomSynth} so that it is
possible for the keys to discriminate between the actions.
%
% Given a
% mapping on actions $f_a$, we need $f_a(c_i)$ to be determined by a
% superset of the same fields as $c_i$. We denote this formally: given
% $\vec k \longrightarrow c_i$ we need
% $\vec {k'} \longrightarrow f_a(c_i)$ , where
% $\vec k \subseteq \vec {k'}$. Further, if we have logical actions
% $ \vec {k_1} \longrightarrow c_i, \vec {k_2} \longrightarrow c_j$,
% with $i \neq j$, and $\vec {k_1} \neq \vec{k_2}$ such that
% $\vec {k_r} \longrightarrow f_a(c_i), f_a(c_j)$, then we need a
% ``separable'' function of type
% $f_k : \vec {k_1} + \vec{k_2} \to \vec{k_r}$. Specifically, we need to
% know that
% \[
%   \not \exists \vec{v_1}, \vec{v_2}. \; f_k^1(\vec{v_1}) = f_k^2(\vec{v_2})
% \]
% in addition to the functional constraint that
% \[
%   \forall \pkt.\; \pkt = \vec{v_i} \Leftrightarrow \pkt = f_k^i(\vec{v_i}), \quad i = 1,2
% \]
%
Specificaly, given constants $h_{i,j}$ which tell us that logical
action $c_i$ is implemented as concrete action $c_j$, we want to
synthesize functions $f : \Expr^n \times \mathbb N \to \Expr^m \times \mathbb N $ and activation
bits $h_{i,j}$, which map logical keys to concrete keys subject to the
following conditions:
\begin{enumerate}[align=left]
% \item[$(\textsc{Discernable})$]
%   $\forall \vec k, i. ~ \pkt \sim \vec {k} \vee i = a \Leftrightarrow
%   (\exists! j. f_{i,j}(\vec {k}) \sim \pkt \vee j = b)$
\item[$(\textsc{Implementable})$] \((\pkt \sim \vec{k} \wedge \wp(c_i, \pkt = \pkt') \Leftrightarrow \pkt \sim f_1(\vec k, i) \wedge \wp(d_{f_2(\vec k, i)}, \pkt = \pkt'))\)
\end{enumerate}


We say the function $f$ is \emph{consistent} if it satisfies these
properties. Then we construct a solution to the problem
$\textsc{EditSynth}(\apply{L, \vec k, \vec c, a}, \apply{R, \vec k,
  \vec d, b})$ as follows
\[ f^c(+,L,\vec v, i) = \{(+,R,\vec u, j) \mid (\vec u,j) = f(\vec v,i) \} \]

Note that \textsc{Disjoint} and \textsc{ActEq} constraints on
$h_{i,j}$ means that $|f^c(\rho)|$ is always 1, so w.l.o.g., we can
consider $f^c$ to have the following type $f^c : \Edit \to \Edit$.

\begin{theorem}
  $f^c$ is a solution to
  $\textsc{EditSynth}()$ with $c_l = \apply{L, \vec k, \vec c, a}$ and $c_r = \apply{R,\vec l,
    \vec d, b})$.
\end{theorem}

\begin{proof}
  Given a logical table instantiation $\tau_l$ and a concrete table
  instantiation $\tau_r$, such that
  $\wp_{\tau_l}(c_l, \pkt = \pkt') \Leftrightarrow \wp_{\tau_r}(c_l,
  \pkt = \pkt')$ and an addition $\rho$ to table $L$, we need to show that
  \[\wp_{\tau_l + \rho}(\apply{L, \vec k, \vec c, a}, \pkt = \pkt') \Leftrightarrow
    \wp_{\tau_r + f^c(\rho)}(\apply{R, \vec l, \vec c, b}, \pkt = \pkt').\]

  We will (equivalently) show that given
  $\bracbb{\apply{L, \vec k, \vec c, a}}_{\tau_l} =
  \bracbb{\apply{R,\vec l, \vec d, b}}_{\tau_r}$, we can prove
  \[\bracbb{\apply{L, \vec k, \vec c, a}}_{\tau_l + \rho} =
    \bracbb{\apply{R,\vec l, \vec d, b}}_{\tau_r + f^c(\rho)}.\]

  Let $\rho = (+, L, \vec v, i)$, and $f^c(\rho) = (+, R, \vec u, j)$,
  such that $h_{i,j} = \TRUE$ and $\vec u = f_{i,j}(\vec v)$.

  Let $\pkt$ be a packet, then we have two cases,
  $\bracbb{\vec k = \vec v}\;\pkt$ is $\TRUE$, or it is $\FALSE$.
  
  First assume that it is $\FALSE$. Then by definition,
  $\bracbb{\apply{L, \vec k, \vec c, a}}_{\tau_l + \rho}\;\pkt =
  \bracbb{\apply{L, \vec k, \vec c, a}}_{\tau_l}\; \pkt$. Since
  $h_{i,j} = 1$, the \textsc{Discernable} condition implies that
  $\bracbb{\vec l = \vec u} = \FALSE$ also, so by definitions we
  compute
  $\bracbb{\apply{R,\vec l, \vec d, b}}_{\tau_r + f^c(\rho)}\; \pkt =
  \bracbb{\apply{R,\vec l, \vec d, b}}_{\tau_r}\; \pkt$. This case is
  finished by the \textsc{EditSynth} assumption.

  Now assume that $\bracbb{\vec k = \vec v}\;\pkt = \TRUE$. Then, we
  reduce
  $\bracbb{\apply{L, \vec k, \vec c, a}}_{\tau_l + \rho}\;\pkt =
  \bracbb{c_i}\;\pkt$, and similarly,
  $\bracbb{\apply{R,\vec l, \vec d, b}}_{\tau_r + f^c(\rho)}\; \pkt =
  \bracbb{d_j}\; \pkt$. Then, the result follows by \textsc{Implementable}.


\end{proof}


We can also produce the other direction of this theorem.

\begin{theorem}
  Given a solution $f$ to $\textsc{EditSynth}$ where
  $c_l = \apply{L, \vec k, \vec c, a}$ and
  $c_r = \apply{R,\vec l, \vec d, b})$, then we can construct an $f^c$
  that is consistent.
\end{theorem}

\begin{proof}
  Given $f$ such that for every $\tau_l, \tau_r, \rho$,
  \[\wp_{\tau_l}(\apply{L, \vec k, \vec c, a}, \pkt = \pkt') \Leftrightarrow
    \wp_{\tau_r}( \apply{R,\vec l, \vec d, b}, \pkt = \pkt')\]
  implies that
  \[\wp_{\tau_r + \rho}(\apply{L, \vec k, \vec c, a}, \pkt = \pkt') \Leftrightarrow
    \wp_{\tau_l + \sum f(\rho)}(\apply{R,\vec l, \vec d, b}, \pkt = \pkt')\]

  We can equivalently say that forall $\pkt$
    \[\bracbb{\apply{L, \vec k, \vec c, a}}_{\tau_l}~\pkt =
    \bracbb{\apply{R,\vec l, \vec d, b}}_{\tau_r}~\pkt\]
  implies that
  \[\bracbb{\apply{L, \vec k, \vec c, a}}_{\tau_r + \rho}~\pkt =
    \bracbb{\apply{R,\vec l, \vec d, b}}_{\tau_l + \sum f(\rho)}~\pkt\]

  Assume there exists

\end{proof}




\subsection{Generalizing : Converting a network program into a single table}

\begin{figure}[pt]
  \[\begin{array}{l >{\triangleq}c l}
      \tablify {x := e} && \apply{\textsf{fresh s}, \cdot, \{x := e\}^*} \\
      \tablify {\assume b} && \apply{\textsf{fresh s}, \cdot, \{\assume b\}^*} \\
      \tablify {\assert b} && \apply{\textsf{fresh s}, \cdot, \{\assert b\}^*} \\
      \tablify {\apply{s, \vec k, \vec c, c_d}} && \apply{s, \vec k, \vec c, c_d} \\
      \tablify {c \angel c'} && \tablify c \diamond \tablify{c'} \\
      \tablify {c ; c'} && \tablify c \circ \tablify{c'}\
    \end{array}\]
  \[
    \begin{array}l
      \apply{s,\vec k, \vec c, a} \circ \apply{t, \vec m, \vec d, b} \\
      \triangleq \apply{st, (\vec x, \vec m), \textsf{map}\; (;)\; (\vec c \times \vec d), a|\vec c| + b}
    \end{array}
  \]
  \[
    \begin{array}l
      \apply{st,\vec k, \vec c, a} \diamond \apply{t, \vec m, \vec d, b} \\
      \triangleq \apply{st, (\vec x, \vec m), \textsf{map}\; (\angel)\; (\vec c \times \vec d), a|\vec c| + b}
    \end{array}
  \]
  
  \caption{Table-normalization function. We star actions as a
    shorthand to denote the default action. The definition of
    $(\circ)$ treats $a$ and $b$ as indices into the vectors $\vec c$
    and $\vec d$. Union on vectors is a uniquifying
    concatenation. $(\times)$ is the cartesian product on vectors, and
    \textsf{map} treats vectors as lists. }
  \label{fig:tablify}
\end{figure}

We can convert a general program into a disjunction of tables using
the function $\tablify c \in \Cmd$, which is defined in
Figure~\ref{fig:tablify}. Assuming the disjunction is deterministic,
our analysis on single tables stands. In fact, allowing disjunctions
makes it easy to compute the functional dependency relation
$(\longrightarrow)$; now $\vec k \longrightarrow c$ in a program $\Angel_{t_i}t_i$,
where each $t_i$ is a table, if there exists a table $t_i$ such that
$\keys{t_i} = \vec k$ and $c \in \actions{t_i}$.

We can also show that for a $c$-complete instantiation function $\tau$
there is always an equivalent instantiation $\tau_t$ for $\tablify c$
(Lemma~\ref{lem:tablifiedinst}), and vice versa
(Lemma~\ref{lem:pretablifiedinst}).

\begin{lemma}
  \label{lem:tablifiedinst}
  For every command $c$ and complete instantiation $\tau$, there
  exists $\tau_t$ such that
  $\bracbb{c}_\tau = \bracbb{\tablify c}_{\tau_t}$.
\end{lemma}

\begin{proof}[Proof Idea]
  Proceed by induction on the structure of $c$:
  \begin{enumerate}[align=left]
  \item[($\apply{s, \vec k, \vec c, a}$)] Immediate.
  \item[($x := e$)] $\bracbb{\tablify {x := e}}_{\langle \rangle} = \bracbb{\apply{t,\cdot, \{x := e\}^*}}_{\langle \rangle} = \bracbb{x:=e}_{\langle \rangle} = \bracbb{x:=e}_{\tau}$.
  \item[($\assert b$)] \textit{sim.}
  \item[($\assume b$)] \textit{sim.}
  \item[($c;c'$)] Our IHs give us $\tau_t$ and $\tau_t'$ such that
    $\bracbb{c}_\tau = \bracbb{\tablify c }_{\tau_t}$ and
    $\bracbb{c'}_\tau = \bracbb{\tablify {c'}}_{\tau_t'}$. Let
    $\apply{s, \vec k, \vec c, d} = \tablify c$, and
    $\apply{s', \vec{k'}, \vec{c'}, d'} = \tablify {c'}$.

    In this limited context $\tau_t$ is a sequence of additions to a
single table: $(e_1,a_1), \ldots, (e_n, a_n)$, and similarly $\tau_t'$
is $(e_1', a_1), \ldots, (e_m', a_m')$.

    We include the row $((e_i,e'_j), (a_i,a_j))$ if $e_1, e_n$
    satisfies the test $\wp(c_{a_i}, \fvs(c_{a_i}) = \vec k')$.

    This proves our goal, since a packet will take rule
    $((e_i,e'_j), (a_i,a_j))$ iff $e_i = \vec{k}$ and
    $\bracbb{c_{a_i}}\; e_j' = \vec{k'}$ by the relationship between
    the semantics and weakest preconditions.
    
  \item[($c \angel c'$)] Our IHs give us $\tau_t$ and $\tau_t'$ such
    that $\bracbb{c}_\tau = \bracbb{\tablify c }_{\tau_t}$ and
    $\bracbb{c'}_\tau = \bracbb{\tablify {c'}}_{\tau_t'}$. Let
    $\apply{s, \vec k, \vec c, d} = \tablify c$, and
    $\apply{s', \vec{k'}, \vec{c'}, d'} = \tablify {c'}$.

    For each row $(\vec e, i) \in \tau_t$ we insert temporary rules
    $(\vec e\vec{\ast'}, i \angel \star)$. Similarly, for every
    $(\vec {e'}, j) \in \tau_t$ we insert temporary row
    $(\vec \ast \vec{e'}, \star \angel j)$. Where the $\ast$ variables
    are unification placeholders. Then, for every pair
    $(\vec e\vec{\ast'}, i \angel \star')$, and $(\vec \ast \vec{e'}, \star
    \angel j)$ we insert a rule $(\vec e\vec e', i \angel j)$ if it it
    is satifiable for $\vec k = \vec e$ and for
    $\vec {k'} = \vec {e'}$. Then we create catch-all rules
    elaborating the wildcards $\ast$ and $\ast'$ setting $\star$ to
    $a$ and $\star'$ to $a'$. Call this $\tau^\ast$.

    We know that
    $\bracbb{ c \angel c'}_{\tau} = \bracbb{c}_\tau \cup \bracbb{c'}_\tau
    = \bracbb{\tablify c}_{\tau_t} \cup
    \bracbb{\tablify{c'}}_{\tau_t'}$, so all we need to show is that
    $\bracbb{\tablify{c}}_{\tau_t} \cup
    \bracbb{\tablify{c'}}_{\tau_t'} = \bracbb{\tablify{c \angel
        c'}}_{\tau^\ast}$.

    Let $\pkt$ be a packet. Assume $\pkt$ matches some rule
    $(\vec{e}, i)$ in $\tau_t$ and another rule $(\vec{e'}, j)$ in
    $\tau_t'$. Then,
    $\bracbb{\tablify{c}}_{\tau_t}\; \pkt \cup
    \bracbb{\tablify{c'}}_{\tau_t'}\;\pkt = \bracbb{c_i}\; pkt \cup
    \bracbb{c'_j}\; \pkt$. Similarly, $\tau^\ast$ has a rule
    $(\vec e\vec e', i \angel j)$, so
    $\bracbb{\tablify{c \angel c'}}_{\tau^\ast} = \bracbb{c_i}\; \pkt
\cup \bracbb{c'_j}\;\pkt$, and we're done.

    Now assume that $\pkt$ takes the default rule in $\tau_t$ and in
    $\tau_t'$, then
    $\bracbb{\tablify{c}}_{\tau_t}\; \pkt \cup
    \bracbb{\tablify{c'}}_{\tau_t'}\;\pkt = \bracbb{c_a}\; pkt \cup
    \bracbb{c'_{a'}}\; \pkt$. By construction, $\tau^*$ will also have
    no rule matching $\pkt$, so we will take its default action
    $c_a \angel c_{a'}$, which has the desired semantics, and we're
    done.

    Now, consider the case that $\pkt$ takes a rule $(\vec{e}, i)$ in
    $\tau_t$, and the default action $a'$ in $\tau_t'$. Now,
    $\bracbb{\tablify{c}}_{\tau_t}\; \pkt \cup
    \bracbb{\tablify{c'}}_{\tau_t'}\;\pkt = \bracbb{c_i}\; \pkt \cup
    \bracbb{c'_{a'}}\; \pkt$. By construction, $\tau^\ast$ has a row
    $(\vec{e}\vec{e'}, i \angel a')$ such that $(\vec{e},i)$ is a
    rule, and packets matching $\vec{e'}$ take the default action in
    $\tau_t'$. So,
    $\bracbb{\tablify{c \angel c'}}_{\tau^\ast} = \bracbb{c_i}\; \pkt
    \cup \bracbb{c'_{a'}}\;\pkt$, and we're done.
    
    
  \end{enumerate}
\end{proof}

\begin{lemma}
  \label{lem:pretablifiedinst}
  For every command $c$ and complete instantiation $\tau_t$, there
  exists $\tau$ such that $\bracbb{\tablify c}_{\tau_t} = \bracbb{c}_\tau$.
\end{lemma}

\begin{proof}
Proceed by induction on the structure of $c$.
\begin{enumerate}[align=left]
  \item[($\apply{s, \vec k, \vec c, a}$)] Immediate.
  \item[($x := e$)] $\bracbb{\tablify {x := e}}_{\langle \rangle} = \bracbb{\apply{t,\cdot, \{x := e\}^*}}_{\langle \rangle} = \bracbb{x:=e}_{\langle \rangle} = \bracbb{x:=e}_{\tau}$.
  \item[($\assert b$)] \textit{sim.}
  \item[($\assume b$)] \textit{sim.}

  \item[($c;c'$)] execute action from first table on packet with keys
    as data values, then match becomes corresponding rule.
  \item[($c \angel c'$)] rule projection is lossy, so -- delete rules
    that go to default action.
\end{enumerate}
\end{proof}

These lemmas (which we've proved by providing constructive witnesses)
together let us perform all of our synthesis by analyzing the single
table and converting our updates back into the original program.

\begin{figure}[pt]
  \newcommand{\linedef}[1]{\multicolumn{3}{>{\qquad}l}{#1}}
  \[
    \begin{array}{lcl}
      \oneify(x:=e, \tau)
      &\triangleq& \langle \rangle \\
      \oneify(\assume b, \tau)
      &\triangleq& \langle \rangle \\
      \oneify(\assert b, \tau)
      &\triangleq& \langle \rangle \\
      \oneify(\apply{t, \vec k, \vec c, a}, \tau)
      &\triangleq&  \tau \\
      \oneify(c;c', \tau)
      &\triangleq& \\
      \linedef{\mathit{let}\; \apply{t, \vec k, \vec c, a} = \tablify{c}} \\
      \linedef{\mathit {let}\; \apply{s, \vec l, \vec d, b} = \tablify {c'}} \\
      \linedef{\mathit {forall}\; (\vec{v_i}, a_i) \in \tau(t),\; (\vec{w_j}, b_j) \in \tau(s)} \\
      \linedef{\{ (\vec{v_i}\vec{u},a_i|\vec c| + b_j) \mid }\\
      \linedef{\qquad \vec{k} = \vec{v_i}\wedge \vec{l} = \vec{u} \Rightarrow \wp(c_{a_i}, \vec l = \vec{w_j})) \}} \\
      \oneify(c \angel c', \tau)
      & \triangleq & \\
      \linedef{\mathit{let}\; \apply{t, \vec k, \vec c, a} = \tablify{c}} \\
      \linedef{\mathit{let}\; \apply{s, \vec l, \vec d, b} = \tablify{c'}} \\
      \linedef{\mathit{forall}\; (\vec{v_i}, a_i) \in \tau(t),\; (\vec{w_j}, b_j) \in \tau(s)} \\
      \linedef{\quad \{(\vec{v_i}\vec{w_j}, a_i|\vec c| +  b_i) \mid \vec{v_i} = \vec k \wedge \vec{w_j} = \vec l\; \mathit{is}\; \textsc{Sat}\} }\\
      \linedef{\quad \cup\{ (\vec{v_i}\vec{u},a_i|\vec c| + b) \mid  \vec u \in \textsf{dom}(\vec l)}\\
      \linedef{\quad\qquad \vec{v_i} = \vec k \wedge \vec{w_j} = \vec l\; \mathit{is}\; \textsc{Unsat}\}} \\
      \linedef{\quad \cup\{ (\vec{u}\vec{w_j}, a|\vec c| +  b_j) \mid  \vec u \in \textsf{dom}(\vec l)}\\
      \linedef{\quad\qquad \vec{v_i} = \vec k \wedge \vec{w_j} = \vec l\; \mathit{is}\; \textsc{Unsat}\}} \\
      \end{array}
  \]
  \caption{$\oneify$ is the function that takes an instantiation on many
    table, to an instantiation on one table.}
\end{figure}

\begin{figure}[pt]
  \newcommand{\linedef}[1]{\multicolumn{3}{>{\qquad}l}{#1}}
  \[
    \begin{array}{lcl}
      \manyify(x:=e, \tau)
      &\triangleq& \langle \rangle \\
      \manyify(\assume b, \tau)
      &\triangleq& \langle \rangle \\
      \manyify(\assert b, \tau)
      &\triangleq& \langle \rangle \\
      \manyify(\apply{t, \vec k, \vec c, a}, \tau)
      &\triangleq&  \tau \\
      \manyify(c;c', \tau)
      &\triangleq& \\
      \linedef{\mathit{let}\; \apply{t, \vec k, \vec c, a} = \tablify{c}} \\
      \linedef{\mathit {let}\; \apply{s, \vec l, \vec d, b} = \tablify {c'}} \\
      \linedef{\displaystyle \bigcup_{(\vec{v_i}\vec{w_j},a_i|\vec c| + b_j) \in \tau(ts)}
      \begin{array}{l} \{(t,\vec{v_i},a_i)\} \;\cup \\ \{(s, \vec u, b_j) \mid \vec l = \vec{u}\Rightarrow \wp^{\vee}(c_{a_i}, \vec{l} = \vec{w_j})\}
        \end{array}} \\
      \manyify(c \angel c', \tau)
      & \triangleq & \\
      \linedef{\mathit{let}\; \apply{t, \vec k, \vec c, a} = \tablify{c}} \\
      \linedef{\mathit{let}\; \apply{s, \vec l, \vec d, b} = \tablify{c'}} \\
      \linedef{\displaystyle \bigcup_{(\vec{v_i}\vec{w_j}, a_i|\vec c| + b) \in \tau(ts)} \{(t, \vec{v_i}, a_i)\}\cup \{(s,\vec{w_j}, b_j)\}}
      \end{array}
  \]
  \caption{$\manyify$ is the function that takes an instantiation on many
    table, to an instantiation on one table.}
\end{figure}



\begin{figure}
  \[
    \begin{array}{lcl}
      \simplify(x:=e, \vec k \longrightarrow_d \vec c)
      &\triangleq& \vec k[x \mapsto e] \longrightarrow_d \overrightarrow{x:=e;c} \\
      \simplify(\assume b, \vec k \longrightarrow_d \vec c)
      & \triangleq& \vec k \longrightarrow_d \overrightarrow{\assume b; c}\\
      \simplify(\assert b, \vec k \longrightarrow_d \vec c)
      & \triangleq& \vec k \longrightarrow_d \overrightarrow{\assert b; c}\\
      \simplify(\vec l \longrightarrow_f \vec b, \vec k \longrightarrow_d \vec c)
      &\triangleq& \\
      \simplify(c;c', \vec k \longrightarrow_d \vec c)
      & \triangleq& \simplify(c, \simplify(c', \vec k \longrightarrow_d \vec c)) \\
    \end{array}
  \]

\end{figure}